\PassOptionsToPackage{unicode=true}{hyperref} % options for packages loaded elsewhere
\PassOptionsToPackage{hyphens}{url}
%
\documentclass[]{article}
\usepackage{lmodern}
\usepackage{amssymb,amsmath}
\usepackage{ifxetex,ifluatex}
\usepackage{fixltx2e} % provides \textsubscript
\ifnum 0\ifxetex 1\fi\ifluatex 1\fi=0 % if pdftex
  \usepackage[T1]{fontenc}
  \usepackage[utf8]{inputenc}
  \usepackage{textcomp} % provides euro and other symbols
\else % if luatex or xelatex
  \usepackage{unicode-math}
  \defaultfontfeatures{Ligatures=TeX,Scale=MatchLowercase}
\fi
% use upquote if available, for straight quotes in verbatim environments
\IfFileExists{upquote.sty}{\usepackage{upquote}}{}
% use microtype if available
\IfFileExists{microtype.sty}{%
\usepackage[]{microtype}
\UseMicrotypeSet[protrusion]{basicmath} % disable protrusion for tt fonts
}{}
\IfFileExists{parskip.sty}{%
\usepackage{parskip}
}{% else
\setlength{\parindent}{0pt}
\setlength{\parskip}{6pt plus 2pt minus 1pt}
}
\usepackage{hyperref}
\hypersetup{
            pdftitle={Visualizing Pangolin Uncertainty (a.k.a. Pangoluncertainty)},
            pdfauthor={Devan Becker},
            pdfborder={0 0 0},
            breaklinks=true}
\urlstyle{same}  % don't use monospace font for urls
\usepackage[margin=1in]{geometry}
\usepackage{color}
\usepackage{fancyvrb}
\newcommand{\VerbBar}{|}
\newcommand{\VERB}{\Verb[commandchars=\\\{\}]}
\DefineVerbatimEnvironment{Highlighting}{Verbatim}{commandchars=\\\{\}}
% Add ',fontsize=\small' for more characters per line
\usepackage{framed}
\definecolor{shadecolor}{RGB}{248,248,248}
\newenvironment{Shaded}{\begin{snugshade}}{\end{snugshade}}
\newcommand{\AlertTok}[1]{\textcolor[rgb]{0.94,0.16,0.16}{#1}}
\newcommand{\AnnotationTok}[1]{\textcolor[rgb]{0.56,0.35,0.01}{\textbf{\textit{#1}}}}
\newcommand{\AttributeTok}[1]{\textcolor[rgb]{0.77,0.63,0.00}{#1}}
\newcommand{\BaseNTok}[1]{\textcolor[rgb]{0.00,0.00,0.81}{#1}}
\newcommand{\BuiltInTok}[1]{#1}
\newcommand{\CharTok}[1]{\textcolor[rgb]{0.31,0.60,0.02}{#1}}
\newcommand{\CommentTok}[1]{\textcolor[rgb]{0.56,0.35,0.01}{\textit{#1}}}
\newcommand{\CommentVarTok}[1]{\textcolor[rgb]{0.56,0.35,0.01}{\textbf{\textit{#1}}}}
\newcommand{\ConstantTok}[1]{\textcolor[rgb]{0.00,0.00,0.00}{#1}}
\newcommand{\ControlFlowTok}[1]{\textcolor[rgb]{0.13,0.29,0.53}{\textbf{#1}}}
\newcommand{\DataTypeTok}[1]{\textcolor[rgb]{0.13,0.29,0.53}{#1}}
\newcommand{\DecValTok}[1]{\textcolor[rgb]{0.00,0.00,0.81}{#1}}
\newcommand{\DocumentationTok}[1]{\textcolor[rgb]{0.56,0.35,0.01}{\textbf{\textit{#1}}}}
\newcommand{\ErrorTok}[1]{\textcolor[rgb]{0.64,0.00,0.00}{\textbf{#1}}}
\newcommand{\ExtensionTok}[1]{#1}
\newcommand{\FloatTok}[1]{\textcolor[rgb]{0.00,0.00,0.81}{#1}}
\newcommand{\FunctionTok}[1]{\textcolor[rgb]{0.00,0.00,0.00}{#1}}
\newcommand{\ImportTok}[1]{#1}
\newcommand{\InformationTok}[1]{\textcolor[rgb]{0.56,0.35,0.01}{\textbf{\textit{#1}}}}
\newcommand{\KeywordTok}[1]{\textcolor[rgb]{0.13,0.29,0.53}{\textbf{#1}}}
\newcommand{\NormalTok}[1]{#1}
\newcommand{\OperatorTok}[1]{\textcolor[rgb]{0.81,0.36,0.00}{\textbf{#1}}}
\newcommand{\OtherTok}[1]{\textcolor[rgb]{0.56,0.35,0.01}{#1}}
\newcommand{\PreprocessorTok}[1]{\textcolor[rgb]{0.56,0.35,0.01}{\textit{#1}}}
\newcommand{\RegionMarkerTok}[1]{#1}
\newcommand{\SpecialCharTok}[1]{\textcolor[rgb]{0.00,0.00,0.00}{#1}}
\newcommand{\SpecialStringTok}[1]{\textcolor[rgb]{0.31,0.60,0.02}{#1}}
\newcommand{\StringTok}[1]{\textcolor[rgb]{0.31,0.60,0.02}{#1}}
\newcommand{\VariableTok}[1]{\textcolor[rgb]{0.00,0.00,0.00}{#1}}
\newcommand{\VerbatimStringTok}[1]{\textcolor[rgb]{0.31,0.60,0.02}{#1}}
\newcommand{\WarningTok}[1]{\textcolor[rgb]{0.56,0.35,0.01}{\textbf{\textit{#1}}}}
\usepackage{graphicx,grffile}
\makeatletter
\def\maxwidth{\ifdim\Gin@nat@width>\linewidth\linewidth\else\Gin@nat@width\fi}
\def\maxheight{\ifdim\Gin@nat@height>\textheight\textheight\else\Gin@nat@height\fi}
\makeatother
% Scale images if necessary, so that they will not overflow the page
% margins by default, and it is still possible to overwrite the defaults
% using explicit options in \includegraphics[width, height, ...]{}
\setkeys{Gin}{width=\maxwidth,height=\maxheight,keepaspectratio}
\setlength{\emergencystretch}{3em}  % prevent overfull lines
\providecommand{\tightlist}{%
  \setlength{\itemsep}{0pt}\setlength{\parskip}{0pt}}
\setcounter{secnumdepth}{0}
% Redefines (sub)paragraphs to behave more like sections
\ifx\paragraph\undefined\else
\let\oldparagraph\paragraph
\renewcommand{\paragraph}[1]{\oldparagraph{#1}\mbox{}}
\fi
\ifx\subparagraph\undefined\else
\let\oldsubparagraph\subparagraph
\renewcommand{\subparagraph}[1]{\oldsubparagraph{#1}\mbox{}}
\fi

% set default figure placement to htbp
\makeatletter
\def\fps@figure{htbp}
\makeatother


\title{Visualizing Pangolin Uncertainty (a.k.a. Pangoluncertainty)}
\author{Devan Becker}
\date{2021-02-10}

\begin{document}
\maketitle

\begin{Shaded}
\begin{Highlighting}[]
\KeywordTok{library}\NormalTok{(here)}
\KeywordTok{library}\NormalTok{(ggplot2)}
\KeywordTok{library}\NormalTok{(dplyr)}
\KeywordTok{library}\NormalTok{(tidyr)}
\KeywordTok{library}\NormalTok{(stringr)}
\end{Highlighting}
\end{Shaded}

\hypertarget{goal}{%
\section{Goal}\label{goal}}

Visualize/summarize the results of sending resampled uncertainty
matrices through pangolin.

\hypertarget{vis-setup}{%
\section{Vis Setup}\label{vis-setup}}

Each row in the dataset represents a single sample from one uncertainty
matrix (only dealing with one accession number for now). The columns are
as folows:

\begin{itemize}
\tightlist
\item
  \texttt{taxon}: the accession name, followed by ``.0'' for the
  consensus sequence and ``0.1'' to ``0.1000'' for the 1,000 samples
  from the matrix.

  \begin{itemize}
  \tightlist
  \item
    This is split into \texttt{taxon}, which is all identical for one
    file, and \texttt{sample}, which is 0 for the conseq and 1:1000 for
    the 1,000 samples.
  \end{itemize}
\item
  \texttt{lineage}: The called lineage from pangolin
\item
  \texttt{probability}: the bootstrap support for this lineage call
\item
  \texttt{pangoLEARN\_version}, \texttt{status}, and \texttt{note}:
  extra info from pangolin
\end{itemize}

\begin{Shaded}
\begin{Highlighting}[]
\NormalTok{unc1 <-}\StringTok{ }\KeywordTok{read.csv}\NormalTok{(}\KeywordTok{here}\NormalTok{(}\StringTok{"data/pangolineages/"}\NormalTok{, }\StringTok{"ERR4085809_pangolineages.csv"}\NormalTok{))}
\NormalTok{unc1 <-}\StringTok{ }\NormalTok{unc1 }\OperatorTok\StringTok{ }
\StringTok{    }\KeywordTok{separate}\NormalTok{(}\DataTypeTok{col =} \StringTok{"taxon"}\NormalTok{, }\DataTypeTok{sep =} \StringTok{"}\CharTok{\textbackslash{}\textbackslash{}}\StringTok{."}\NormalTok{, }
        \DataTypeTok{into =} \KeywordTok{c}\NormalTok{(}\StringTok{"taxon"}\NormalTok{, }\StringTok{"sample"}\NormalTok{)) }\OperatorTok\StringTok{ }
\StringTok{    }\KeywordTok{mutate}\NormalTok{(}\DataTypeTok{taxon =} \KeywordTok{str_replace}\NormalTok{(taxon, }\StringTok{"}\CharTok{\textbackslash{}\textbackslash{}}\StringTok{_"}\NormalTok{, }\StringTok{""}\NormalTok{))}
\KeywordTok{head}\NormalTok{(unc1)}
\end{Highlighting}
\end{Shaded}

\begin{verbatim}
##        taxon sample lineage probability pangoLEARN_version    status note
## 1 ERR4085809      0     B.1           1         2021-02-21 passed_qc   NA
## 2 ERR4085809      1     A.2           1         2021-02-21 passed_qc   NA
## 3 ERR4085809      2       B           1         2021-02-21 passed_qc   NA
## 4 ERR4085809      3       B           1         2021-02-21 passed_qc   NA
## 5 ERR4085809      4  B.1.98           1         2021-02-21 passed_qc   NA
## 6 ERR4085809      5       B           1         2021-02-21 passed_qc   NA
\end{verbatim}

To prep the data, I calculate another column called ``prop'', which
represents the number of samples that were assigned the sam lineage as
the one in that row. So, for a lineage assigned B.1, prop would be the
total number of B.1s in the sampled lineages divided by 1000.

\begin{Shaded}
\begin{Highlighting}[]
\NormalTok{unc2 <-}\StringTok{ }\NormalTok{unc1 }\OperatorTok
\StringTok{    }\KeywordTok{group_by}\NormalTok{(lineage) }\OperatorTok
\StringTok{    }\KeywordTok{summarise}\NormalTok{(}\DataTypeTok{prop =} \KeywordTok{n}\NormalTok{() }\OperatorTok{/}\StringTok{ }\NormalTok{(}\KeywordTok{nrow}\NormalTok{(unc1) }\OperatorTok{-}\StringTok{ }\DecValTok{1}\NormalTok{)) }\OperatorTok
\StringTok{    }\KeywordTok{right_join}\NormalTok{(unc1, }\DataTypeTok{by =} \StringTok{"lineage"}\NormalTok{) }\OperatorTok
\StringTok{    }\KeywordTok{arrange}\NormalTok{(}\KeywordTok{as.numeric}\NormalTok{(sample))}
\end{Highlighting}
\end{Shaded}

\begin{verbatim}
## `summarise()` ungrouping output (override with `.groups` argument)
\end{verbatim}

\begin{Shaded}
\begin{Highlighting}[]
\KeywordTok{head}\NormalTok{(unc2)}
\end{Highlighting}
\end{Shaded}

\begin{verbatim}
## # A tibble: 6 x 8
##   lineage    prop taxon     sample probability pangoLEARN_version status   note 
##   <chr>     <dbl> <chr>     <chr>        <dbl> <chr>              <chr>    <lgl>
## 1 B.1     0.186   ERR40858~ 0                1 2021-02-21         passed_~ NA   
## 2 A.2     0.00694 ERR40858~ 1                1 2021-02-21         passed_~ NA   
## 3 B       0.204   ERR40858~ 2                1 2021-02-21         passed_~ NA   
## 4 B       0.204   ERR40858~ 3                1 2021-02-21         passed_~ NA   
## 5 B.1.98  0.00231 ERR40858~ 4                1 2021-02-21         passed_~ NA   
## 6 B       0.204   ERR40858~ 5                1 2021-02-21         passed_~ NA
\end{verbatim}

Now we can view the differences between the pangolin bootstrap support
and the actual proportion of samples with that lineage designation!

\begin{Shaded}
\begin{Highlighting}[]
\KeywordTok{ggplot}\NormalTok{(unc2) }\OperatorTok{+}\StringTok{ }
\StringTok{    }\KeywordTok{aes}\NormalTok{(}\DataTypeTok{x =}\NormalTok{ prop, }\DataTypeTok{y =}\NormalTok{ probability) }\OperatorTok{+}\StringTok{ }
\StringTok{    }\KeywordTok{geom_point}\NormalTok{()}
\end{Highlighting}
\end{Shaded}

\includegraphics{PangoVis_files/figure-latex/unnamed-chunk-3-1.pdf}

Right now it's not very interesting because all of the pangolin
probabilities are 1. I changed something in my code, and I don't know
why this broke it. It's on my TODO.

\end{document}
