\documentclass[12pt]{article}
\usepackage{geometry}                
\geometry{letterpaper}                   % ... or a4paper or a5paper or ... 
\usepackage[parfill]{parskip}    % Activate to begin paragraphs with an empty line rather than an indent
\usepackage{graphicx}
\usepackage{amssymb}
\usepackage{xcolor}


\title{Sequencing Uncertainty Propagation}
\author{David Champredon}
\date{\today}                                           % Activate to display a given date or no date

\newcommand{\comment}[1]{\textsl{\textcolor{cyan}{#1}}}
\newcommand{\sq}[1]{\texttt{\textcolor{brown}{#1}}}

\begin{document}
\maketitle

\section{Background}

Identifying the sequence of nucleotides from a biological sample is a complex process which is fraught with noise. 

Assuming the biological sample of interest has been properly isolated (that is it is complete, has no damages or contamination), the ``Next-Generation Sequencing'' (NGS) usually involves:
\begin{itemize}
\item fragment DNA in small pieces (length $<$ 400bp)
\item making numerous copies of the small DNA fragments (amplification)
\item sequencing, i.e., identifying the nucleotides from a fluorescent tag attached
\item putting back the small pieces together by aligning them (de novo) or mapping them on benchmark libraries
\end{itemize}

For Sanger sequencing, we have"
\begin{itemize}
\item a
\item b
\end{itemize}


Errors can be introduced at each of these steps for various reasons \cite{Beerenwinkel:2011}. It is probably not feasible to determine what is the source of the noise, nor to try to eliminate it completely.
The goal here is to acknowledge there is uncertainty in the output sequence given from any sequencing method and to propose a method to propagate this uncertainty in any downstream analysis.
Currently, this uncertainty is recognized and even quantified with sequencing quality scores (FASTQ files), but it does not seem those scores are used to inform a probabilistic model to represent the sequence. 
Simply put, we shouldn't treat the result of sequencing as a \emph{certainty}.

\section{Probabilistic representation}
 
We can represent probabilistically a nucleotide sequence in a matrix form. 


 

%\subsection{}



\end{document}  